\documentclass{llncs}
\usepackage{hyperref}
\begin{document}
\title{Comparison between the Twitter Search Network and Facebook Like/Comment
Network for two Movies: The Hobbit and The Interview}
\author{Georgiana Diana Ciocirdel}
\institute{Polytechnic University of Bucharest, Romania}
\maketitle
%
\begin{abstract}
We are hereby analysing four different networks, from two social media portals,
Twitter and Facebook. The networks are related to the movies
\href{http://www.thehobbit.com/}{The Hobbit} and
\href{http://www.theinterview-movie.com/}{The Interview}. The movie The Hobbit
first appeared on the 12/01/2014 in the UK and in the month that followed in the
rest of the world. The movie ‘The Interview’ is quite controversial and has
faced many issues before being released solely in the US, first on the
12/11/2014 and then during Christmas time. We have run our analysis on data
collected around the release dates. The Twitter networks we are analysing are
directed, with edges linking users that have authored a post to users that have
retweeted it. The Facebook networks are being analysed in three consecutive
steps, marking important release dates around the world of the two movies; these
are unimodal, undirected networks, with edges linking users that have liked or
commented on the same post made by the official Facebook pages of the two
movies. The analysis reveals a few important things about the four networks: the
two Twitter networks differ a lot between them - although big, The Interview's
Twitter graph is a "young" one, meaning that authorities and hubs haven't yet
been formed properly, the posts are sparse and with very few users retweeting
more than one post. In comparison, The Hobbit's Twitter graph has well defined
authorities and most of its users have retweeted more than one popular tweet.
The Facebook networks also differ: The Hobbit has more users liking and
commenting on their posts, especially around the US release, while The Interview
network is smaller and doesn't change much in time.
\end{abstract}
%
\section{Collecting the data}
In order to collect the necessary data for our study, we first tried to popular
data retrieving tools, such as Wolfram Mathematica, 
%
\end{document}
